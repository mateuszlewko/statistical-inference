\documentclass[a4paper]{article}

\usepackage{fullpage}
\usepackage{parskip}
% \usepackage{tikz} 
\usepackage{amsmath}
\usepackage{hyperref}
\usepackage[T1]{fontenc}
\usepackage[polish]{babel}
\usepackage[utf8]{inputenc}

\title{Test Wilcoxona}
\author{Mateusz Lewko}
\date{\today}

\begin{document}

\maketitle

\section{Wstęp}

 Test sumy rang Wilcoxona służy do sprawdzenia, czy wartości próbek pobranych z dwóch niezależnych populacji, są jednakowo duże. Innymi słowy, pozwala sprawdzić, czy dwie, losowo wybrane próbki, zostały wybrane z populacji o takim samym rozkładzie. 

\subsection{Koniecznie założenia}
\begin{enumerate}
    \item Wszystkie obserwacje z obydwu grup są od siebie niezależne.
    \item Obserwacje możemy porównać i uporządkować.
    \item Hipoteza zerowa: obserwacje pochodzą z populacji o takim samym rozkładzie.
    \item Hipoteza alternatywna: rozkłady populacji są różne.
\end{enumerate}

\section{Metoda obliczania}
\subsection{Opis testu}
Po przeprowadzeniu eksperymentu dla dwóch grup osobników, otrzymaliśmy odpowiednio $n_1$ i $n_2$ pomiarów. Chcemy sprawdzić hipotezę, że rozkłady wyników dla obu populacji są takie same ($H_0 : G_1 = G_2$). Test Wilcoxona wykryje przesunięcie $G_1$ względem $G_2$. Gdy podejrzewamy, że grupa pierwsza jest przesunięta w prawo względem grupy pierwszej, to hipoteza alternatywna przyjmuje postać $H_a: G_1 < G_2$. W ogólnym przypadku, gdy nie podejrzewamy żadnego konkretnego kierunku przesunięcia, to  $H_a: G_1 \neq G_2$.

Test Wilcoxona bazuje na uporządkowaniu wszystkich $n = n_1 + n_2$ obserwacji. Każda obserwacja otrzymuje rangę: najmniejsza ma rangę równą $1$, a największa rangę $n$. Statystyką testową $W$ będzie suma rang obserwacji dla jednej z grup. Niech 
\begin{align}
    w_1 = \text{suma rang obserwacji z grupy pierwszej.}
\end{align}

\subsection{Przykład obliczania rang}
Po uporządkowaniu obserwacji z dwóch grup otrzymaliśmy następującą kolejność:
\begin{center}
   \begin{tabular}{ | l | l | l | l | l | l | l | }
    \hline
    Number grupy & 1 & 1 & 2 & 2 & 1 & 2 \\ \hline
    Ranga        & 1 & 2 & 3 & 4 & 5 & 6 \\ \hline
    \end{tabular}
\end{center}

Wtedy $w_1 = 1 + 2 + 5 = 8$ oraz $w_2 = 3 + 4 + 6 = 13$.

\subsection{Rozkład testu rang}
W celu wyznaczenia \textit{wartości\_p}, należy znać rozkład prawdopodobieństwa możliwych sum rang. Będziemy wtedy w stanie określić, jak prawdopodobne jest otrzymanie sumy rang jednej z grup, przy założeniu, że grupy pochodzą z tego samego rozkładu. Jeżeli obserwacje pochodzą z tych samych rozkładów, to każda ich kolejność (permutacja) jest tak samo prawdopodobna. To rozumowanie nasuwa następujący sposób na wyznaczenie gęstości sum rang:
\begin{enumerate}
    \item Wyznaczyć wszystkie permutacje elementów z jednej grupy.
    \item Dla każdej permutacji obliczyć sumę jej rang. 
    \item Częstość występowania danej sumy to prawdopodobieństwo jej otrzymania. Prawdopodobieństwo rang $r$ i $n_1 + n_2 - r$ są takie same -- gęstość jest symetryczna.

\end{enumerate}

\subsection{Obliczanie \textit{wartości\_p}}
Załóżmy, że $H_a : G_1 > G_2$, wtedy oczekiwalibyśmy, że suma rang w grupie drugiej będzie nadzwyczaj duża, zgodnie z rozkładem sum rang. Stąd \textit{wartość\_p} to 
\begin{align}
     \textit{wartości\_p} = P(W \geq w_1)
\end{align}
W przypadku, gdy spodziewamy się, że $H_a : G_1 < G_2$, to \textit{wartość\_p} obliczamy analogicznie.

Dal dwustronnego testu z $H_a : G_1 \neq G_2$, \textit{wartość\_p} przyjmuje postać

\begin{align}
     \textit{wartości\_p} = 2 * P(W \geq w_1) \\
     \text{lub}\\
     \textit{wartości\_p} = 2 * P(W \leq w_1)
\end{align}

w zależności od tego, w którym kwantylu się znajdziemy.

\section{Przybliżanie rozkładem normalnym}

Jeżeli $n_1$ i $n_2$ są większe od $10$, to statystyka $W$ ma w przybliżaniu rozkład $N(\mu_1, \sigma_1)$, gdzie
\begin{align}
    \mu_1 = \frac{n_1 (n_1 + n_2 + 1)}{2} \\
    \sigma_1 = \sqrt{\frac{n_1 n_2(n_1+n_2+1)}{12}}
\end{align}

\subsection{Przykład}

Dane są grupy o licznościach $n_1 = 10$, $n_2 = 12$. Chcemy policzyć $P(W_1 \geq 145)$. Wiemy że

\begin{align}
    P(W_1 \geq w_a) \approx P(Z \geq z) \\ 
    z = \frac{w_1 - \mu_1}{\sigma_1} \\
    Z \sim N(0, 1)
\end{align}

Wtedy, $\mu_1 = 10 \times (10 + 12 + 1)/ 2 = 115$ oraz 
\begin{align}
    \sigma_1 = \sqrt{\frac{10 \times 12 \times (10 + 12 + 1)}{12}} = 15.16575 \\
    P(W_1 \geq 145) \approx P \Big(Z \geq \frac{145-115}{15.16575}\Big) = P(Z \geq 1.978) = 0.024
\end{align}

\subsection{Uporządkowanie zawierające remisy}

W przypadku występowania remisów po uporządkowaniu obserwacji, wzór na $\sigma_1$ jest następujący

\begin{align}
    \sigma_1 = \sqrt{\frac{n_1 n_2}{12} \bigg( (n+1) - \sum_{i=1}^{k}{\frac{t_i^{3} -t_i}{n(n-1)}} \bigg)}
\end{align}

gdzie $n = n_1 + n_2$, $t_i$ to liczba różnych obserwacji z rangą $i$, a $k$ to liczba różnych rang.

\begin{thebibliography}{9}
\bibitem{ch10} \url{https://www.stat.auckland.ac.nz/~wild/ChanceEnc/Ch10.wilcoxon.pdf}
\bibitem{wiki} \url{https://en.wikipedia.org/wiki/Mann%E2%80%93Whitney_U_test?oldformat=true}
\end{thebibliography}

\end{document}