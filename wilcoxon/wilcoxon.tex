\documentclass[a4paper]{article}

\usepackage{fullpage} % Package to use full page
\usepackage{parskip} % Package to tweak paragraph skipping
\usepackage{tikz} % Package for drawing
\usepackage{amsmath}
\usepackage{hyperref}
\usepackage[T1]{fontenc}
\usepackage[polish]{babel}
\usepackage[utf8]{inputenc}

\title{Test Wilcoxona}
\author{Mateusz Lewko}
\date{\today}

\begin{document}

\maketitle

\section{Wstęp}

 Test sumy rang Wilcoxona służy do sprawdzenia, czy wartości próbek pobranych z dwóch niezależnych populacji są jednakowo duże. Innymi słowy, pozwala sprawdzić czy, dwie, losowo wybrane próbki zostały wybrane z populacji o takim samym rozkładzie. 

\subsection{Koniecznie założenia}
\begin{enumerate}
    \item Wszystkie obserwacje z obydwu grup są od siebie niezależne.
    \item Obserwacje możemy porównać i uporządkować.
    \item Hipoteza zerowa: obserwacje pochodzą z populacji o takim samym rozkładzie.
    \item Hipoteza alternatywna: rozkłady populacji są różne.
\end{enumerate}

\section{Metoda obliczania}

The definition of the limit of $f(x)$ at $x=a$ denoted as $f'(a)$ is:

\begin{equation}
f'(a) = \lim_{h\to0}\frac{f(a+h)-f(a)}{h}
\end{equation}

The following code can be used in sage to give the above limit:


\begin{align}
y=&ax+b&&\text{(definition of a straight line)}\nonumber\\
  &f'(a)x+b&&\text{(definition of the derivative)}\nonumber\\
  &f'(a)x+f(a)-f'(a)a&&\text{(we know that the line intersects $f$ at $(a,f(a))$}\nonumber
\end{align}

\section{Przybliżanie rozkładem normalnym}


\end{document}